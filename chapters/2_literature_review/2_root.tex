\chapter{Literature Review}
\label{chap::literature_review}

This is the section describing the related work and outlining the research gap that your document is addressing.

This is an example of an ``in-text'' citation: \cite{jordan2015machine}.

Remember to insert each entry to the \textit{references.bib} file.

\textbf{Do not} describe concepts that have limited connection to your goal. Whoever is going to read this thesis is meant to be an ``expert''!

Use schemas or figures if necessary: they help a lot in understanding. If you take the schemas from previous work, do cite them in the caption. Figures can be cited with \textbackslash ref command such as Figure \ref{fig::example_picture}.

\begin{figure}[!htbp]
    \centering
    \includegraphics[width=0.75\columnwidth]{images/figures/example_figure.jpg}
    \caption{The caption of figures goes BELOW the figure.}
    \label{fig::example_picture}
\end{figure}

All acronyms should be defined in \textit{acronym.tex} and referred to with \textbackslash ac command. The first referred acronym will display the whole name such as \ac{ML}, and only abbreviations will be displayed afterwards such as \ac{ML}.